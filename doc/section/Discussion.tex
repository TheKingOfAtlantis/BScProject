% !TeX root = ../main.tex
\documentclass[../main.tex]{subfile}
\begin{document} \section{Discussion}
    The RF1:RF2 hypothesis suggested that the principle driving stop codon selection is their co-evolution with the  ratio of RF1 \& RF2 expression and genomic GC content. The overall results of this study, strongly indicate that the paradoxical underutilisation of TAG is likely not driven by RF1:RF2 expression as first through but instead being driven by an unknown process shared across all domains of life.

    While the usage of stop codons in bacterial protein coding sequences was recapitulated using the phylogenetically controlled dataset, asserting more strongly that the effect occurs. The implied consequences of the RF1:RF2 hypothesis in anything but primary stop codons of bacterial protein coding sequences were not.
    The study was able to re-establish previous results by \textcite{Ho2019}, as shown in \cref{fig:results/bacteria/protein/shifted}, the relative usage of stop codons downstream of bacterial protein coding sequences, appear to be under similar selection pressures as the primary stop codon. The RF1:RF2 hypothesis would predict that usage of TAG would not be supressed in this context, instead following a similar trend to TGA (i.e. being strongly correlated with GC content), as it is not connected to the translational machinery.
    The continued suppression of TAG usage means either a) selection against TAG codons is not driven by RF1:RF2 expression, or b) there is selection for additional +1 downstream stop codons to mediate fail-safe termination of translation -- this could additionally be mediated by the potential +4T nucleotide enrichment of the primary stop codon \autocite{Ho2019} by the terminational machinery, meaning RF1:RF2 expression could still hold. However, \textcite{Ho2019} had also shown that positive selection for additional stop codons is not likely occur in bacteria, indicating that it is more likely due to selection by a non-RF1:RF2 expression mechanism.

    Exploring the relative usage of TAA, TGA and TAG along tRNA sequences in bacteria, showed that GC content does not affect the relative usage of the stop codons in tRNA. Instead all three codons remained relatively constant for all GC content values. The GC-independent nature of their usage, likely arises from a strong pressure to maintain their structural and functional integrity (given how critical they are to normal cellular function). As such the pattern seen, likely arises due to stronger pressure to maintain these "stop codons" regardless of the GC pressure on them. Interestingly, the usage of TAG was greater that TAA or TGA, given that protein coding sequences show the reverse, it may potentially be an indication of some connection between the usage of these codons in both contexts, where the high utilisation of TAG in one context supresses it in the other. Ultimately, the usage of stop codons in tRNA seems to present rather inconclusive evidence with relation to the RF1:RF2 hypothesis.

    The relative primary stop codon usage in archaeal and human protein coding sequences, however, presented a similar pattern of TAG suppression to what was found in bacteria. This indicates that a similar process is directing the usage of stop codons in archaea and humans, since they utilise a similar class I release factors. Given that both domains utilise a single class I RF to bind all three stop codons, and that the RF1:RF2 hypothesis would predict that TAG usage would be similar to TGA usage (in both archaea and eukaryotic), the underutilisation of TAG presents a very stronger indication that the RF1:RF2 hypothesis may not hold true, as it cannot explain why the pattern persists across the domains of life where RF1 and RF2 are not present.

    Furthermore, the archaeal genomes also shows similar +1 downstream codon usage in all three reading frames, as seen in bacteria, which again would not be predicated by the RF1:RF2 hypothesis.
    These patterns in archaea and humans simply by virtue of being found in a different domain which doesn't utilise RF1 and RF2, strongly suggests that the RF1:RF2 hypothesis does not explain the underlying mechanism in bacteria.

    Since the relative expression of RF1 and RF2 likely does not dictate the usage of each stop codon even in bacteria, another unknown process is likely driving the selection in all domains of life. However, while the RF1:RF2 hypothesis conclusions appears to be incorrect, their results still indicated correlation between stop codon usage and RF1 and RF2 expression. Instead then, the direction of causality may flow in the other direction, with the unknown process dictating the relative usage of each stop codon across all domain, which then dictates the levels of RF1 and RF2 expression in bacteria rather than the other way round. The ratio of RF1:RF2 expression in bacteria could potentially arises from the combination of increased expression of RF1 and RF2 to avoid readthrough errors, and with suppression of RF1 and RF2 due to the pressure associated energetic cost of synthesis. Over time the ratio of RF1 \& RF2 expression would equilibrate as comes to match the relative abundance of the stop codons.

    \subsection{Limitations}
        While attempts have been made to limit oversampling of similar sequences in the dataset, much greater control could be incorporated to further reduce the overrepresentation of duplicated genes in the dataset. This could be achieved via BLAST, to find duplicates or highly homologous sequences, from which a single sequence could be selected from to remove duplicated sequences for the analysis.
        Additionally, identifying and potentially normalising for the underlying factor driving some TAG points in the archaeal dataset to be shifted noticeably higher (higher than TGA usage in some cases), would not produce much cleaner results but could also be insightful.
        Furthermore, the archaeal dataset was significantly smaller than the bacterial genome.
    \subsubsection{Future Directions}
        Exploring the usage of stop codons in organisms which utilise translation table 4 could provide an informative comparison with the organisms analysed. Rather than using the canonical TAA, TGA, and TAG stop codons, as explored so far, these organisms use TGA to encode the amino acid tryptophan, as such the only stop codons utilised are TAA and TAG \autocite{translationTable}. Understanding the behaviour of stop codons in these organisms could provide critical insights into the underlying selection pressures acting on these stop codons (especially if TAG is still selected against), as TAC is the only GC-rich stop codon available. Additionally, exploring the usage of TAC could identify if the selection is due to the presence of guanine in the third position, or if it is more generally connected the GC3 content of the stop codon.

    % --------------------------------------------------------

    % The results suggest that RF1:RF2 expression is not driving the selection of stop codons in bacteria.
    % The emergence of the same relative stop codon utilisation downstream in bacterial protein coding sequences, indicates that the usage of stop codons potentially could be driven by another process.

    % Isolating the influence of the two release factors entirely by exploring the usage of stop codons in archaea and human (eukaryotic) genomes, showed that while they have a subtle positive correlation with GC3, the usage with respect to TGA is much lower

    % The usage of TAG in archaeal protein coding sequences is lower than TGA usage, but unlike in bacteria their was a positive correlation with GC3 content. A proper evaluation of the connection between GC and TAG usage in archaeal protein coding sequences, could be achieved by testing the data against a Monte Carlo simulation of GC dependent codon usage, a statistically significant difference would strength the argument for a pressure to replace TAG existing even in archaea.



    % % Human Genome
    % % Much greater control could be incorporate to avoid duplicated gene sequences for affecting the results
    % % Could be achieved through running BLAST to identify and isolate duplicates

    % % RF1:RF2 Hypothesis
    % While relative expression of RF1 and RF2 likely does not dictate the usage of each stop codon (even in bacteria), instead some other unknown process is driving the stop codon selection. However, while the conclusion drawn by \autocite{PLACEHOLDER} appears to be incorrect, their results still had indicated some sort of correlation. Instead then, the direction of causality which was suggested may be the other way around, with the relative usage of each stop codons dictating the levels of RF1 and RF2 expression rather than the other way round - a potential mechanism for this could that bacteria would need to produce a sufficient quantity of the release factors to avoid readthroughs, however producing too much has an energetic cost associated with it, so the expression of each release factor would then tend over time to match the relative abundance of their target codon (excluding TAA).

    % Exploring the usage of stop codons in organisms which utilise translation table 4 could provide an informative comparison with the organisms analysed. Rather than using the canonical TAA, TGA, and TAG stop codons, as explored so far, these organisms uses TGA to encode the amino acid trypotophan and so the only stop codons utilised are TAA and TAG \autocite{translationTable}. Understanding the behaviour of stop codons in these organisms could provide critical insights into the underlying selection pressures acting on stop codons, as TAG is the only GC-rich stop codon available.

    % tRNA under selection for stability
\end{document}
