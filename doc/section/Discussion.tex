% !TeX root = ../main.tex
\documentclass[../main.tex]{subfile}
\begin{document} \section{Discussion}


    % RF1:RF2 Hypothesis
    The relative expression of RF1 and RF2 clearly does not dictate the usage of each stop codon (even in bacteria), instead some other unknown process is driving the stop codon selection. However, while the conclusion drawn by \autocite{PLACEHOLDER} appears to be incorrect, their results still had indicated some sort of correlation. Instead then, the direction of causality which was suggested may be the other way around, with the relative usage of each stop codons dictating the levels of RF1 and RF2 expression rather than the other way round - a potential mechanism for this could that bacteria would need to produce a sufficient quantity of the release factors to avoid readthroughs, however producing too much has an energetic cost associated with it, so the expression of each release factor would then tend over time to match the relative abundance of their target codon (excluding TAA).

\end{document}
