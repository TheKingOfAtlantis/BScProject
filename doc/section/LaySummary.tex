% !TeX root = ../main.tex
\documentclass[../main.tex]{subfile}
\begin{document}
    \section*{Lay Summary}

    The translation of messenger RNA (mRNA) to proteins is mediated by the ribosome, which clamps down on the start of the mRNA and moves along, reading in steps of three nucleotides (called codons). Each codon, which is composed of a specific combination of nucleotides, codes for a specific amino acid. In many organisms, three of these codons (TAA,TGA,TAG) tell the ribosome to stop reading the mRNA, releasing the protein it is has built based on the mRNA.

    The proteins which help the ribosome recognise these stop codons (called release factors or RFs) are different in each domain of life (bacteria, archaea and eukaryotes). Bacteria use two RFs to recognise these stop codons (RF1 and RF2), each recognises a different pair of codons (one UAA/UGA and the other UAA/UAG), while archaea and eukaryotes only need one release factor to recognise all three stop signals.

    The usage of stop codons depends on multiple factors including the GC content of the genome -- which measures the percentage of the genome that is G \& C nucleotides. Genomes with lower GC content have more TAA as it has no G or C nucleotides, while genomes with high GC content have more TGA stop codons. It was noticed in bacteria that TAG usage did not change with GC (even thought its like TGA) and that it was also underutilised compared with TAA and TGA, after some studies it was hypothesised that the amount of RF1 and RF2 might effect be effecting the usage of stop codons -- the RF1:RF2 hypothesis.

    By exploring the usage of TAA, TGA and TAG in regions of the bacterial genome just after the stop codon, which are not involved translation, as well as looking at its usage in archaea and eukaryotes. It was possible to show that TAG was being underutilised, in all of these places, even though the RF1:RF2 hypothesis would predicate the TAG would behave like TGA. As such it is likely that the RF1:RF2 hypothesis is not true and that a universal mechanism is supressing the usage of TAG.
\end{document}
